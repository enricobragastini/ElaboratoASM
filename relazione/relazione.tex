\documentclass[a4paper,11pt,oneside]{book}

% PACCHETTI
\usepackage{hyperref}           % hyperlinks
\usepackage{tabto}              % strumento per inserire tab nel testo
\usepackage[                    % geometria della pagina
    a4paper,
    inner=2cm,
    outer=3cm,
    top=3cm,
    bottom=3cm,
    bindingoffset=1.2cm,
    headheight=14pt
]{geometry}
\usepackage[utf8]{inputenc}     % 3 pacchetti per l'italiano
\usepackage[italian]{babel}
\usepackage[T1]{fontenc}
\usepackage{titlesec}           % custom chapter titles

\usepackage{fancyhdr}
\usepackage{multicol}
\usepackage[arrowdel]{physics} 
\usepackage{amsmath}
\usepackage{tikz}

\usepackage{graphicx}           % IMMAGINI
\graphicspath{ {./images/} }
\usepackage{wrapfig}

\usepackage{csquotes}
\usepackage{caption}

\usepackage{listings}
%\usepackage{minted}

% INFORMAZIONI SUL DOCUMENTO
\title{
    \Large{\textbf{Elaborato Assembly}} \\
    Architettura degli Elaboratori      \\
    A.A. 2020/2021 - Corso di Laurea in Informatica
    }
%\author{Enrico Bragastini - VR456374 \\ Davide Bianchini - VR456697 \\ Andrea Mafficini - VRxxxxxx}
\author{
  Enrico Bragastini\\
  \texttt{VR456374}
  \and
  Davide Bianchini \\
  \texttt{VR456697}
  \and
  Andrea Mafficini \\
  \texttt{VR462441}
}
\titleformat{\chapter}[display]{\normalfont\bfseries}{}{0pt}{\LARGE}
\date{}

% aggiugere logo e migliorare lo stile


% CONTENUTO
\begin{document}
\pagestyle{fancy}
\fancyhf{}
\rhead{}
\lhead{\nouppercase\leftmark}
\cfoot{\thepage}
\frontmatter

% Prima pagina - Titolo
\maketitle
\tableofcontents

\mainmatter
\chapter{Descrizione dell'elaborato}
La \textbf{notazione polacca inversa} (reverse polish notation, \textbf{RPN}) è una notazione per la scrittura di espressioni aritmetiche in cui gli operatori binari, 
anziché utilizzare la tradizionale notazione infissa, usano quella postfissa; ad esempio, l’espressione $5 + 2$ in RPN verrebbe scritta $5 \; 2 \; +$. La \textbf{RPN} è 
particolarmente utile perché \emph{non necessita dell’utilizzo di parentesi}. 

Si intende realizzare un programma in assembly che, letta in input una stringa rappresentante un’espressione ben formata in \textbf{RPN}, scriva in output il risultato ottenuto dalla valutazione dell’espressione.
Per il calcolo di un'espressione in \textbf{RPN} si considerano solamente gli operatori fondamentali: somma, sottrazione, moltiplicazione e divisione.
Nel caso in cui l'espressione contenga caratteri diversi da numeri o da operatori, verrà restituita come output la stringa \verb|Invalid|.

Il codice sorgente \emph{main.c} fa una chiamata a una funzione \emph{extern} chiamata \emph{postfix} scritta in assembly. Questa funzione riceve come parametri due puntatori ai relativi array
che rappresentano la stringa di input e la stringa di output. La funzione si occuperà quindi di leggere la stringa di input, elaborare il risultato e scriverlo nell'array di output.
La lettura e la scrittura su file vengono gestite dal \emph{main.c}.

~\newline
\underline{Esempio:}
\begin{lstlisting}[language=Bash, showstringspaces=false]
    $ echo "100 10 - 10 * -4 /" > in_1.txt
    $ ./postfix
    $ cat out_1.txt 
      -225
\end{lstlisting}
L'espressione postfissa ``100 10 - 10 * -4 /'' corrisponde all'espressione in notazione infissa $(100-10) * 10 / (-4)$, che dà come risultato $-225$.

\end{document}