\documentclass[a4paper,11pt,oneside]{book}

% PACCHETTI
\usepackage{hyperref}           % hyperlinks
\usepackage{tabto}              % strumento per inserire tab nel testo
\usepackage[                    % geometria della pagina
    a4paper,
    inner=2cm,
    outer=3cm,
    top=3cm,
    bottom=3cm,
    bindingoffset=1.2cm,
    headheight=14pt
]{geometry}
\usepackage[utf8]{inputenc}     % 3 pacchetti per l'italiano
\usepackage[italian]{babel}
\usepackage[T1]{fontenc}

\usepackage{fancyhdr}
\usepackage{multicol}
\usepackage[arrowdel]{physics} 
\usepackage{amsmath}
\usepackage{tikz}

\usepackage{graphicx}           % IMMAGINI
\graphicspath{ {./images/} }
\usepackage{wrapfig}

\usepackage{csquotes}
\usepackage{caption}

\usepackage{listings}

\usepackage{titlesec}
\titleformat{\chapter}[display]
{\normalfont\huge\bfseries}{\chaptertitlename\ \thechapter}{20pt}{\Huge}
\titlespacing*{\chapter}{0pt}{0pt}{40pt}

% INFORMAZIONI SUL DOCUMENTO
\title{
    {\includegraphics{logo-univr.png}} \\
    \LARGE{\textbf{Elaborato Assembly}} \\
    Architettura degli Elaboratori      \\
    A.A. 2020/2021 - Corso di Laurea in Informatica
}
%\author{Enrico Bragastini - VR456374 \\ Davide Bianchini - VR456697 \\ Andrea Mafficini - VRxxxxxx}
\author{
  Enrico Bragastini\\
  \texttt{VR456374}
  \and
  Davide Bianchini \\
  \texttt{VR456697}
  \and
  Andrea Mafficini \\
  \texttt{VR462441}
}
\titleformat{\chapter}[display]{\normalfont\bfseries}{}{0pt}{\LARGE}
\date{}

% aggiugere logo e migliorare lo stile


% CONTENUTO
\begin{document}
\pagestyle{fancy}
\fancyhf{}
\rhead{}
\lhead{\nouppercase\leftmark}
\cfoot{\thepage}
\frontmatter

% Prima pagina - Titolo
\maketitle
\tableofcontents

\mainmatter
\chapter{Descrizione dell'elaborato}
La \textbf{notazione polacca inversa} (reverse polish notation, \textbf{RPN}) è una notazione per la scrittura di espressioni aritmetiche in cui gli operatori binari, 
anziché utilizzare la tradizionale notazione infissa, usano quella postfissa; ad esempio, l’espressione $5 + 2$ in RPN verrebbe scritta $5 \; 2 \; +$. La \textbf{RPN} è 
particolarmente utile perché \emph{non necessita dell’utilizzo di parentesi}. 

Si intende realizzare un programma in assembly che, letta in input una stringa rappresentante un’espressione ben formata in \textbf{RPN}, scriva in output il risultato ottenuto dalla valutazione dell’espressione.
Per il calcolo di un'espressione in \textbf{RPN} si considerano solamente gli operatori fondamentali: somma, sottrazione, moltiplicazione e divisione.
Nel caso in cui l'espressione contenga caratteri diversi da numeri o da operatori, verrà restituita come output la stringa \verb|Invalid|.

Il codice sorgente \emph{main.c} fa una chiamata a una funzione \emph{extern} chiamata \emph{postfix} scritta in assembly. Questa funzione riceve come parametri due puntatori ai relativi array
che rappresentano la stringa di input e la stringa di output. La funzione si occuperà quindi di leggere la stringa di input, elaborare il risultato e scriverlo nell'array di output.
La lettura e la scrittura su file vengono gestite dal \emph{main.c}.

~\newline
\underline{Esempio:}
\begin{lstlisting}[language=Bash, showstringspaces=false]
    $ echo "100 10 - 10 * -4 /" > in_1.txt
    $ ./postfix
    $ cat out_1.txt 
      -225
\end{lstlisting}
L'espressione postfissa ``100 10 - 10 * -4 /'' corrisponde all'espressione in notazione infissa $(100-10) * 10 / (-4)$, che dà come risultato $-225$.


\chapter{Metodo di lavoro}

Abbiamo deciso di lavorare sempre in gruppo, senza dividere il progetto in parti a cui lavorare singolarmente.
Per comodità, abbiamo usufruito della piattaforma Zoom per lavorare tutti e tre contemporaneamente al progetto.

Inoltre, per condividere il codice ed evitare spiacevoli inconvenienti, abbiamo fatto uso di Git e GitHub.

~\newline
Inizialmente, la necessità era quella di avere una visione generale del codice, abbiamo quindi creato varie etichette, per suddividere il programma.
In ogni sezione abbiamo scritto delle \emph{pseudo-istruzioni} per descrivere cosa avrebbe fatto quella determinata parte di codice assembly.

Una volta terminata questa fase iniziale di \emph{brainstorming}, abbiamo implementato e commentato uno ad uno il codice di ogni etichetta, testandone il corretto funzionamento 
prima di passare alla etichetta successiva.


\chapter{Codice Assembly}

\section{Scelte progettuali e struttura del codice}
Abbiamo scelto di non suddividere il codice in file multipli o di utilizzare ulteriori funzioni in quanto non ne abbiamo sentita la necessità. Il codice
è stato ben suddiviso in sezioni utilizzando le \textbf{etichette}. Inoltre è ben commentato e non supera le 270 righe, per cui è facilmente leggibile e comprensibile. 

~\newline
Il sorgente contenuto nel file \emph{main.c} fa una chiamata alla funzione \emph{postfix} passandole come argomenti i puntatori agli array di input e output.
Questi puntatori, che si trovano nello stack, vengono copiati nei registri \verb|%esi| e \verb|%edi|.

Ciclicamente vengono letti uno alla volta tutti i caratteri della stringa in input. In base al carattere trovato vengono svolte determinate azioni:
\begin{itemize}
  \item \textbf{CIFRA}: Quando viene trovata una cifra, questa viene inserita nella variabile \emph{buffer}. Verrà fatta la push del buffer una volta trovato lo spazio che segue 
  l'ultima cifra di un numero.

  \item \textbf{SPAZIO}: Quando viene trovato uno spazio, è necessario valutare se questo si presenta dopo una cifra o dopo un operatore. Nel primo caso indica la fine di un numero,
  si procederà quindi al push del buffer. Nel secondo caso nessuna operazione è necessaria, si procederà alla lettura del carattere successivo.

  \item \textbf{SOMMA}: Quando viene trovato il carattere di somma, viene fatta la pop dei due operandi, che devono essere sommati e il risultato viene messo in pila.

  \item \textbf{TRATTINO}: Quando viene trovato il trattino, è necessario fare delle considerazioni perché potrebbe indicare l'inizio di un numero negativo oppure l'operatore di sottrazione.
  Viene quindi valutato il carattere successivo, se quest'ultimo è uno spazio viene eseguita la sottrazione in maniera analoga alla somma. Altrimenti si pone la 
  variabile \emph{FLAG} a $-1$ utilizzata al momento della push del buffer, per negare il numero.

  \item \textbf{MOLTIPLICAZIONE}: Quando viene trovato l'asterisco, viene fatta la pop dei due operandi, che devono essere moltiplicati. Il risultato viene messo in pila.

  \item \textbf{DIVISIONE}: Quando viene trovato lo slash, ci si assicura inizialmente che il divisore sia diverso da zero e il dividendo sia maggiore di zero. Se queste condizioni
  sono verificate, viene fatta la pop dei due operandi, viene eseguita la divisione e il risultato viene messo in pila.

  \item \textbf{INVALID}: Il riscontro di un carattere invalido è conseguenza di tutti i controlli eseguiti precedentemente. Quando un carattere non risulta essere 
  una cifra, un operatore o uno spazio, per forza sarà un carattere invalido. Verrà quindi stampata la stringa \verb|Invalid|.
\end{itemize}

\section{Variabili}
Le variabili utilizzate sono quattro e tutte posizionate nella sezione \emph{.data}

\begin{itemize}
  \item \textbf{buffer}, di tipo \emph{int}, rappresenta il buffer in cui vengono salvate le cifre del numero che si sta leggendo. Il suo valore di default è 0 e viene riportata a tale 
  valore alla fine della lettura di ogni numero.

  \item \textbf{flag}, di tipo \emph{int}, rappresenta il segno del numero che si sta leggendo. Il suo valore di default è 1, ovvero numero positivo, viene settata a -1 ogni qualvolta si andrà
  a leggere un numero negativo.

  \item \textbf{invalid\_string}, di tipo \emph{ascii}, rappresenta la stringa da scrivere nella stringa output nel caso vengano trovati dei caratteri invalidi nella stringa di input.
  
  \item \textbf{invalid\_string\_len}, di tipo \emph{long}, rappresenta la lunghezza della stringa contenuta nella precedente variabile.
\end{itemize}

\section{Diagramma di flusso}
\begin{figure}[h]
  \caption{Example of a parametric plot ($\sin (x), \cos(x), x$)}
  \centering
  \includegraphics[width=0.5\textwidth]{spiral}
  \end{figure}


\end{document}